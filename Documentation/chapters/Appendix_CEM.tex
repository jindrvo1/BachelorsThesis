\chapter{Cross-entropy method}
\label{ch:cross_entropy_method}
The cross-entropy method is a Monte Carlo approach to solving optimization problems and rare-event simulations. It was introduced by \citet{RubinsteinCrossEntropyMethodCombinatorial1999} as an extension to his earlier work focused on variance minimization methods for rare-event probability estimation.

The method is based on an iterative procedure which consists of two phases:

\begin{enumerate}
\item Generate a random data sample according to specified attributes.
\item Update the attributes according to best results of data generated in the first step.
\end{enumerate}

\examplespace
\begin{example}
An appropriate example can be found in the application on soccer data. In order to find the best parameters in \eqref{eq:expected_score_altered}, the initial parameters are set to the Elo default values as seen in \eqref{eq:expected_score}.

In the first phase, numerous two-dimensional normal distributions are randomly generated. In the second phase, the log-likelihood errors are calculated using the parameteres given by said distributions and afterwards, the initial attributes are updated based on the best results (i.e. lowest errors).

Note that the algorithm is not limited to be used with normal distribution, nor log-likelihood loss. The methods are to be chosen accordingly to the task's nature. 
\end{example}

To obtain a better picture of the cross-entropy method implementation, a pseudocode for continuous optimization is provided:

\begin{algorithm}[H]
\begin{algorithmic}
\State $v_0 \gets u$
\Comment{Assign initial parameters} 
\State $T \gets 0$
\Comment{Number of iterations}
\State $t \gets 0$
\Comment{Iteration counter}
\While{t < T}
	\State Generate a sample $X_1, \cdots, X_N$ from the density $f(\cdot, v_t)$
	\State Compute the fitnesses of $X_1, \cdots, X_N$
	\State Recognize $n < N$ best results from $X_1, \cdots, X_N$
	\State Set $v_{t+1}$ according to the results recognized in previous step
	\State $t = t + 1$
\EndWhile
\State \Return $v_t$
\end{algorithmic}
\end{algorithm}